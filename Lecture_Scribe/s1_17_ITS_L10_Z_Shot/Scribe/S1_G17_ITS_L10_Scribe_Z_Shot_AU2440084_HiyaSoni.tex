\documentclass[11pt]{article}

\usepackage{amsmath,amssymb,amsthm}
\usepackage{geometry}
\usepackage{setspace}
\usepackage{hyperref}

\geometry{margin=1in}
\setstretch{1.15}

\newtheorem{definition}{Definition}
\newtheorem{theorem}{Theorem}
\newtheorem{proposition}{Proposition}
\newtheorem{algorithm}{Algorithm}

\begin{document}

\begin{center}
\textbf{CSE 400: Fundamentals of Probability in Computing} \\
\textbf{Lecture 10: Randomized Min-Cut Algorithm} \\
Group No.: S1 G17 \\
Date: February 5, 2026
\end{center}

\section{Min-Cut Problem}

\subsection{Why Use Min-Cut}

Min-cut algorithms are used in applications related to network connectivity, reliability, and optimization. They are applied in:
\begin{itemize}
    \item Network design to improve communication efficiency and optimize network flow by identifying the minimum capacity cut.
    \item Communication networks to analyze vulnerability to failures and build robust, fault-tolerant systems.
    \item VLSI design for partitioning circuits into smaller components to reduce interconnectivity complexity.
\end{itemize}

\subsection{What is Min-Cut}

\begin{definition}[Cut-Set]
A cut-set in a graph is a set of edges whose removal breaks the graph into two or more connected components.
\end{definition}

\begin{definition}[Minimum Cut]
Given a graph $G=(V,E)$ with $n$ vertices, the minimum cut (min-cut) problem is to find a minimum cardinality cut-set in $G$.
\end{definition}

Min-cut algorithms such as Karger’s algorithm are randomized and can be sensitive to the initial choice of edges. If critical edges are contracted early, the algorithm may find a smaller cut.

\subsection{Edge Contraction}

The main operation in the min-cut algorithm is \emph{edge contraction}.

\begin{definition}[Edge Contraction]
In contracting an edge $(u,v)$, the two vertices $u$ and $v$ are merged into one vertex, all edges connecting $u$ and $v$ are eliminated, and all other edges are retained. The resulting graph may contain parallel edges but no self-loops.
\end{definition}

\subsection{Successful Min-Cut Run}

A successful min-cut run refers to the success in the outcome of an algorithm designed to find the minimum cut in a graph.

\subsection{Unsuccessful Min-Cut Run}

An unsuccessful min-cut run refers to an iteration of a min-cut algorithm where the algorithm fails to correctly identify the minimum cut of a given graph.

\section{Max-Flow Min-Cut Theorem}

\begin{theorem}[Max-Flow Min-Cut Theorem]
In a flow network, the maximum amount of flow passing from the source to the sink is equal to the total weight of the edges in a minimum cut.
\end{theorem}

\subsection{Related Definitions}

\begin{itemize}
    \item Capacity of a cut: The sum of the capacities of the edges in the cut that are oriented from a vertex in set $X$ to a vertex in set $Y$.
    \item Minimum cut: The cut in the network with the smallest possible capacity.
    \item Minimum cut capacity: The capacity of the minimum cut.
    \item Maximum flow: The largest possible flow from source $S$ to sink $T$.
\end{itemize}

\section{Deterministic Min-Cut Algorithm}

\subsection{Stoer--Wagner Min-Cut Algorithm}

\begin{proposition}
Let $s$ and $t$ be two vertices of a graph $G$. Let $G/\{s,t\}$ be the graph obtained by merging $s$ and $t$. A minimum cut of $G$ is the smaller of:
\begin{itemize}
    \item a minimum $s$--$t$ cut of $G$, and
    \item a minimum cut of $G/\{s,t\}$.
\end{itemize}
\end{proposition}

\subsection{Pseudocode}

\begin{algorithm}[MinimumCutPhase$(G,a)$]
\begin{enumerate}
    \item $A \leftarrow \{a\}$
    \item while $A \neq V$ do
    \begin{itemize}
        \item add to $A$ the most tightly connected vertex
    \end{itemize}
    \item return the cut weight as the cut of the phase
\end{enumerate}
\end{algorithm}

\begin{algorithm}[MinimumCut$(G)$]
\begin{enumerate}
    \item while $|V| \geq 1$ do
    \begin{itemize}
        \item choose any $a \in V$
        \item MinimumCutPhase$(G,a)$
        \item if the cut-of-the-phase is lighter than the current minimum cut, store it
        \item shrink $G$ by merging the two vertices added last
    \end{itemize}
    \item return the minimum cut
\end{enumerate}
\end{algorithm}

\section{Randomized Min-Cut Algorithm}

\subsection{Why Randomized Algorithms}

Randomized algorithms provide a probabilistic guarantee of success and can give a more accurate estimate of the minimum cut with fewer iterations. Properties include efficiency, parallelization, approximation guarantees, avoidance of worst-case instances, heuristic nature, and robustness.

\subsection{Karger’s Randomized Algorithm}

Karger’s algorithm repeatedly contracts randomly selected edges until only two vertices remain. The remaining edges form a cut.

\subsection{Pseudocode}

\begin{algorithm}[Recursive-Randomized-Min-Cut$(G,\alpha)$]
\begin{enumerate}
    \item Input: Undirected multigraph $G$ with $n$ vertices, integer constant $\alpha > 0$
    \item Output: A cut $C$ of $G$
    \item If $n \leq 3$, compute a min-cut of $G$ using brute force and set $C$
    \item Else:
    \begin{itemize}
        \item for $i = 1$ to $\alpha$:
        \begin{itemize}
            \item $G' \leftarrow$ graph obtained by applying $n - \lceil n/\sqrt{\alpha} \rceil$ random contractions to $G$
            \item $C' \leftarrow$ Recursive-Randomized-Min-Cut$(G',\alpha)$
            \item if $i = 1$ or $|C'| < |C|$, set $C \leftarrow C'$
        \end{itemize}
    \end{itemize}
    \item return $C$
\end{enumerate}
\end{algorithm}

\section{Comparison: Deterministic vs Randomized Min-Cut}

The choice of algorithm depends on the specific problem.

\subsection{Deterministic Min-Cut}
\begin{itemize}
    \item Always guarantees an exact minimum cut
    \item Higher time complexity for large graphs
    \item Stoer--Wagner algorithm time complexity: $O(V \cdot E + V^2 \log V)$
\end{itemize}

\subsection{Randomized Min-Cut}
\begin{itemize}
    \item Provides an approximate minimum cut with high probability
    \item Karger’s algorithm time complexity: $O(V^2)$
\end{itemize}

\section{Theorem for Min-Cut Set}

\begin{theorem}
The randomized min-cut algorithm outputs a min-cut set with probability at least $\dfrac{2}{n(n-1)}$.
\end{theorem}

\end{document}
