\documentclass[11pt]{article}
\usepackage[a4paper,margin=1in]{geometry}
\usepackage{amsmath,amssymb}
\usepackage{enumitem}
\usepackage{fancyhdr}

\pagestyle{fancy}
\fancyhf{}
\lhead{CSE 400: Fundamentals of Probability in Computing}
\rhead{Lecture 10 Scribe Report}
\cfoot{\thepage}

\title{
\normalsize School of Engineering and Applied Science (SEAS), Ahmedabad University\\
\vspace{0.2cm}
\textbf{CSE 400: Fundamentals of Probability in Computing}\\
\Large Lecture 10 Scribe: Randomized Min-Cut Algorithm
}
\author{}
\date{}

\begin{document}
\maketitle
\vspace{-2cm}

\begin{center}
\begin{tabular}{ll}
\textbf{Group No.:} & {S1 G17 \hspace{2.5in}} \\[1.5ex]
\textbf{Domain:} & {ITS \hspace{2.5in}} \\[1.5ex]
\textbf{Date:} & {February 5, 2026 \hspace{2.2in}}
\end{tabular}
\end{center}

\hrule
\vspace{0.5cm}

\section*{Min-Cut Problem}

\subsection*{Cut-Set and Min-Cut Definitions}

A cut-set in a graph is a set of edges whose removal disconnects the graph into two or more connected components.

Given an undirected graph $G=(V,E)$ with $|V|=n$, the minimum cut problem is defined as the task of finding a cut-set of minimum cardinality in $G$.

\subsection*{Edge Contraction Mechanism}

Edge contraction is the primary operation used in randomized min-cut algorithms.

Contracting an edge $(u,v)$ consists of:
\begin{itemize}[leftmargin=1.2cm]
\item Merging vertices $u$ and $v$ into a single supernode.
\item Removing all edges directly connecting $u$ and $v$.
\item Retaining all remaining edges.
\end{itemize}

The resulting graph may contain parallel edges and does not contain self-loops.

\section*{Successful Min-Cut Run}

A successful min-cut run is defined as an execution of the algorithm that returns a cut-set whose cardinality equals the minimum cut of the graph.

Node contraction sequence corresponding to the successful run:
\begin{enumerate}[leftmargin=1.2cm]
\item Nodes $3$ and $4$ are merged into supernode $\{3,4\}$.
\item Nodes $2$ and $\{3,4\}$ are merged into supernode $\{2,3,4\}$.
\item Nodes $1$ and $\{2,3,4\}$ are merged into supernode $\{1,2,3,4\}$.
\item The final remaining partition corresponds to the minimum cut.
\end{enumerate}

\section*{Unsuccessful Min-Cut Run}

An unsuccessful min-cut run refers to an execution in which the algorithm contracts at least one edge belonging to a minimum cut, resulting in a cut-set with cardinality strictly larger than the minimum cut.

Node contraction sequence corresponding to the unsuccessful run:
\begin{enumerate}[leftmargin=1.2cm]
\item Nodes $1$ and $2$ are merged into supernode $\{1,2\}$.
\item Nodes $3$ and $4$ are merged into supernode $\{3,4\}$.
\item Nodes $\{1,2\}$ and $\{3,4\}$ are merged into a single supernode.
\item The resulting cut is not minimum.
\end{enumerate}

\section*{Max-Flow Min-Cut Theorem}

The Max-Flow Min-Cut Theorem states:

In a flow network, the maximum amount of flow from a source vertex $S$ to a sink vertex $T$ is equal to the total capacity of a minimum cut separating $S$ and $T$.

Formal definitions:
\begin{itemize}[leftmargin=1.2cm]
\item Capacity of a cut: the sum of capacities of edges directed from partition $X$ to partition $Y$.
\item Minimum cut: a cut with minimum possible capacity.
\item Maximum flow: the largest feasible flow from $S$ to $T$.
\end{itemize}

\section*{Deterministic Min-Cut: Stoer--Wagner Algorithm}

Let $s$ and $t$ be two vertices of a graph $G$.

Let $G/\{s,t\}$ denote the graph obtained by merging vertices $s$ and $t$.

A minimum cut of $G$ is obtained as the minimum of:
\begin{itemize}[leftmargin=1.2cm]
\item A minimum $s$-$t$ cut of $G$.
\item A minimum cut of $G/\{s,t\}$.
\end{itemize}

\subsection*{Algorithm 1: MinimumCutPhase$(G,a)$}

\begin{itemize}[leftmargin=1.2cm]
\item Initialize $A \leftarrow \{a\}$.
\item While $A \neq V$, add the most tightly connected vertex to $A$.
\item Return the cut weight of the phase.
\end{itemize}

\subsection*{Algorithm 2: MinimumCut$(G)$}

\begin{itemize}[leftmargin=1.2cm]
\item While $|V| \geq 1$:
\item Select any vertex $a \in V$.
\item Execute MinimumCutPhase$(G,a)$.
\item Update the current minimum cut if the phase cut is lighter.
\item Shrink $G$ by merging the last two added vertices.
\item Return the minimum cut.
\end{itemize}

\section*{Randomized Min-Cut Algorithm}

\subsection*{Karger’s Randomized Algorithm}

The algorithm repeatedly contracts randomly chosen edges until only two supernodes remain.

The edges between the two remaining supernodes form a cut.

\subsection*{Algorithm 3: Recursive-Randomized-Min-Cut$(G,\alpha)$}

Input: Undirected multigraph $G$ with $n$ vertices and integer $\alpha > 0$.

Output: A cut $C$ of $G$.

\begin{itemize}[leftmargin=1.2cm]
\item If $n \leq 3$:
\begin{itemize}
\item Compute a min-cut of $G$ using exhaustive search.
\end{itemize}
\item Else:
\begin{itemize}
\item For $i = 1$ to $\alpha$:
\item Obtain $G'$ by performing $n - \lceil \frac{n}{\sqrt{\alpha}} \rceil$ random contractions on $G$.
\item Recursively compute $C' =$ Recursive-Randomized-Min-Cut$(G',\alpha)$.
\item If $i=1$ or $|C'| < |C|$, update $C \leftarrow C'$.
\end{itemize}
\item Return $C$.
\end{itemize}

The base case $n \leq 3$ applies only in the context specified on Slide 35.

\section*{Theorem: Success Probability}

For a graph with $n$ vertices, Karger’s randomized min-cut algorithm outputs a minimum cut with probability at least:
\[
\frac{2}{n(n-1)}.
\]

\section*{Complexity Comparison}

\begin{itemize}[leftmargin=1.2cm]
\item Stoer--Wagner deterministic min-cut algorithm runs in time $O(VE + V^2 \log V)$.
\item Karger’s randomized min-cut algorithm runs in time $O(V^2)$ per execution.
\end{itemize}

The randomized algorithm does not scale to the same complexity as $O(VE + V^2 \log V)$ and remains asymptotically distinct.

\end{document}
