\documentclass[11pt]{article}

% ===================== PACKAGES =====================
\usepackage[a4paper,margin=1in]{geometry}
\usepackage{amsmath,amssymb,amsthm}
\usepackage{enumitem}
\usepackage{fancyhdr}
\usepackage{xcolor}
\usepackage{array}
\usepackage{booktabs}

% ===================== HEADER & FOOTER =====================
\pagestyle{fancy}
\fancyhf{}
\lhead{CSE 400: Fundamentals of Probability in Computing}
\rhead{Lecture 10 Scribe}
\cfoot{\thepage}

% ===================== TITLE =====================
\title{
    \normalsize School of Engineering and Applied Science (SEAS), Ahmedabad University\\
    \vspace{0.2cm}
    \textbf{CSE 400: Fundamentals of Probability in Computing}\\
    \Large Lecture 10 Scribe
}
\author{}
\date{}

\begin{document}
\maketitle

\vspace{-2cm}
\begin{center}
    \begin{tabular}{ll}
        \textbf{Group No.:} & S1 G17 \\[1ex]
        \textbf{Name:} & ARYA PATEL \\[1ex]
        \textbf{ID:} & AU2440250 \\[1ex]
        \textbf{Date:} & February 5, 2026
    \end{tabular}
\end{center}

\hrule
\vspace{0.5cm}

\section*{Lecture 10: Randomized Min-Cut Algorithm}

\section{Outline}
\begin{itemize}
    \item Min-Cut Problem
    \item Why use min-cut?
    \item What is min-cut?
    \item Successful min-cut run
    \item Unsuccessful min-cut run
    \item Max-Flow Min-Cut Theorem
    \item Deterministic Min-Cut Algorithm
    \item Stoer--Wagner Minimum Cut Algorithm
    \item Pseudocode
    \item Randomized Min-Cut Algorithm
    \item Why randomized algorithm?
    \item Karger’s Randomized Algorithm
    \item Pseudocode
    \item Comparison: Deterministic vs Randomized Min-Cut
    \item Theorem for min-cut set
    \item Python Simulation
\end{itemize}

\section{Min-Cut Problem}

\subsection{Why use min-cut?}
We use the min-cut algorithm in various applications to solve problems related to network connectivity, reliability, and optimization.

\begin{itemize}
    \item \textbf{Network Design:} Min cut helps in improving the efficiency of communication and optimizing network flow. The algorithm is used in network design to find the minimum capacity cut.
    \item \textbf{Communication Networks:} For understanding the vulnerability of networks to failures, minimum cut can be useful. It helps in building robust and fault-tolerant communication networks.
    \item \textbf{VLSI Design:} In Very Large Scale Integration (VLSI) design, the algorithm is useful for partitioning circuits into smaller components leading to reduced interconnectivity complexity.
\end{itemize}

\subsection{What is min-cut?}

\begin{definition}[Cut-Set]
A cut-set in a graph is a set of edges whose removal breaks the graph into two or more connected components.
\end{definition}

\begin{definition}[Minimum Cut]
Given a graph $G=(V,E)$ with $n$ vertices, the minimum cut (min-cut) problem is to find a cut-set of minimum cardinality in $G$.
\end{definition}

Min-cut algorithms such as Karger’s algorithm are random and can be sensitive to the initial choice of edges.

\subsection{Edge Contraction}

\begin{definition}[Edge Contraction]
The main operation in the algorithm is edge contraction, which removes an edge $(u,v)$ from the graph while simultaneously merging vertices $u$ and $v$ into a single vertex.
\end{definition}

\begin{itemize}
    \item Vertices $u$ and $v$ are merged into one vertex.
    \item All edges connecting $u$ and $v$ are eliminated.
    \item All other edges are retained.
    \item The resulting graph may contain parallel edges but no self-loops.
\end{itemize}

\section{Successful Min-Cut Run}

A successful min-cut run refers to the success in the outcome of an algorithm designed to find the minimum cut in a graph.

\subsection*{Graph Contraction Sequence (Successful Run)}

\begin{center}
\begin{tabular}{c|c|c}
\toprule
Step & Operation & Resulting Vertex Sets \\
\midrule
Initial & Original graph & $\{1,2,3,4\}$ \\
1 & Contract edge $(2,3)$ & $\{1,\{2,3\},4\}$ \\
2 & Contract edge $(\{2,3\},4)$ & $\{1,\{2,3,4\}\}$ \\
\bottomrule
\end{tabular}
\end{center}

The remaining parallel edges between the two supernodes define the minimum cut.

\section{Unsuccessful Min-Cut Run}

An unsuccessful min-cut run refers to an iteration of a min-cut algorithm where the algorithm fails to correctly identify the minimum cut of a given graph.

\subsection*{Graph Contraction Sequence (Unsuccessful Run)}

\begin{center}
\begin{tabular}{c|c|c}
\toprule
Step & Operation & Resulting Vertex Sets \\
\midrule
Initial & Original graph & $\{1,2,3,4\}$ \\
1 & Contract critical edge & $\{\{1,2\},3,4\}$ \\
2 & Contract another edge & $\{\{1,2,3\},4\}$ \\
\bottomrule
\end{tabular}
\end{center}

The final cut does not correspond to the minimum cut of the original graph.

\section{Max-Flow Min-Cut Theorem}

\begin{theorem}[Max-Flow Min-Cut Theorem]
In a flow network, the maximum amount of flow passing from the source to the sink is equal to the total weight of the edges in a minimum cut.
\end{theorem}

\subsection*{Definitions}

\begin{itemize}
    \item \textbf{Capacity of a cut:} Sum of capacities of edges oriented from $X$ to $Y$.
    \item \textbf{Minimum cut:} Cut with smallest possible capacity.
    \item \textbf{Maximum flow:} Largest possible flow from source $S$ to sink $T$.
\end{itemize}

\section{Deterministic Min-Cut Algorithm}

\subsection{Stoer--Wagner Min-Cut Algorithm}

Let $s$ and $t$ be two vertices of a graph $G$. Let $G/\{s,t\}$ denote the graph obtained by merging $s$ and $t$.

\begin{theorem}
A minimum cut of $G$ is the smaller of:
\begin{itemize}
    \item A minimum $s$-$t$ cut of $G$
    \item A minimum cut of $G/\{s,t\}$
\end{itemize}
\end{theorem}

\subsection*{Pseudocode}

\begin{verbatim}
Algorithm 1: MinimumCutPhase(G, a)
A ← {a}
while A ≠ V do
    add to A the most tightly connected vertex
return cut weight

Algorithm 2: MinimumCut(G)
while |V| ≥ 1 do
    choose any a from V
    MinimumCutPhase(G, a)
    if cut-of-the-phase < current minimum then
        store cut
    shrink G by merging last two vertices
return minimum cut
\end{verbatim}

\section{Randomized Min-Cut Algorithm}

\subsection{Why Randomized Algorithm?}

\begin{itemize}
    \item Probabilistic guarantee of success
    \item Fewer iterations
    \item Efficiency
    \item Parallelization
    \item Approximation guarantees
    \item Avoidance of worst-case instances
    \item Robustness
\end{itemize}

\subsection{Karger’s Randomized Algorithm}

\subsection*{Worked Example (Step-by-Step)}

\begin{itemize}
    \item Start with undirected multigraph $G=(V,E)$
    \item Randomly select an edge
    \item Contract the selected edge
    \item Repeat until only two vertices remain
    \item Remaining parallel edges define the cut
\end{itemize}

\subsection*{Pseudocode}

\begin{verbatim}
Algorithm: Recursive-Randomized-Min-Cut(G, α)
if n ≤ α then
    return brute-force min-cut
else
    for i = 1 to α do
        G' ← apply n - n/√α contractions
        C' ← Recursive-Randomized-Min-Cut(G', α)
        if |C'| < |C| then
            C ← C'
return C
\end{verbatim}

\section{Theorem for Min-Cut Set}

\begin{theorem}
The algorithm outputs a minimum cut with probability at least
\[
\frac{2}{n(n-1)}.
\]
\end{theorem}

\subsection*{Proof (As Presented)}

At each contraction step, the probability of not contracting a minimum cut edge is:
\[
1 - \frac{\lambda}{|E|} \geq 1 - \frac{2}{n}.
\]

Thus, the probability that no minimum cut edge is contracted over $(n-2)$ steps is:
\[
\prod_{i=0}^{n-3} \left(1 - \frac{2}{n-i}\right)
= \frac{2}{n(n-1)}.
\]

\section{Comparison: Deterministic vs Randomized Min-Cut}

\begin{center}
\begin{tabular}{p{6cm}p{6cm}}
\toprule
Deterministic Min-Cut & Randomized Min-Cut \\
\midrule
Exact minimum cut & Approximate with high probability \\
Higher time complexity & Lower expected time \\
$O(VE + V^2 \log V)$ & $O(V^2)$ \\
\bottomrule
\end{tabular}
\end{center}

\section{Python Simulation}

\begin{itemize}
    \item Students instructed to open Campuswire post for Lecture 10
    \item Download the provided \texttt{.ipynb} file
\end{itemize}

\hrule
\vspace{0.2cm}
\begin{center}
\small End of Lecture 10 Scribe
\end{center}

\end{document}
